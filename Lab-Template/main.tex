\documentclass[9pt,a4paper,twocolumn,twoside]{lab-class/lab}
\usepackage[spanish,es-noshorthands]{babel}

%----------------------------------------------------------
% TITLE
%----------------------------------------------------------

\journalname{Plantilla de Ejemplo}
\title{Escribiendo un reporte de laboratorio o artículo académico con la clase lab \LaTeX}

%----------------------------------------------------------
% AUTHORS, AFFILIATIONS AND PROFESSOR
%----------------------------------------------------------

\author[a,1]{Author One}
\author[b,2]{Author Two}
\author[b,c,3]{Author Three}

%----------------------------------------------------------

\affil[a]{Affiliation of author one}
\affil[b]{Affiliation of author two}
\affil[c]{Affiliation of author three}

\professor{Nombre del Profesor}

%----------------------------------------------------------
% FOOTER INFORMATION
%----------------------------------------------------------

\institution{Nombre de la Institución}
\footinfo{\LaTeX\ Template}
\theday{26 de Julio, 2024}
\leadauthor{Apellido del Autor et al.}
\course{Creative Commons CC BY 4.0}

%----------------------------------------------------------
% ABSTRACT AND KEYWORDS
%----------------------------------------------------------

\begin{abstract}    
    Bienvenido a la clase lab \LaTeX\ diseñada especialmente para sus reportes de laboratorio o artículos académicos. En esta plantilla de ejemplo, le guiaremos a través del proceso de uso y personalización de este documento según sus necesidades. Para más información sobre esta clase, consulte la sección de apéndices. Allí encontrará códigos que definen aspectos clave de la plantilla, permitiéndole explorar y modificarlos.
\end{abstract}

%----------------------------------------------------------

\keywords{clase \LaTeX, reporte de laboratorio, artículo académico, clase lab}

%----------------------------------------------------------

\begin{document}
    \maketitle 
    \thispagestyle{firststyle} 
    \lababstract 
    % \tableofcontents
    % \linenumbers 
    
%----------------------------------------------------------
% SECTIONS
%----------------------------------------------------------

\section{Introducción}

    \labstart{B}ienvenido a \textit{lab class} para preparar sus reportes de laboratorio o artículos académicos. A lo largo de esta guía, le mostraremos cómo usar esta plantilla y cómo realizar modificaciones a esta clase.
    
    Esta clase incluye los siguientes archivos ubicados en la carpeta 'lab-class': lab.cls, labenvs.sty, labbabel.sty y README.md. Además, main.tex, lab.bib y algunos ejemplos.

\section{Título}

    El comando \verb*|\maketitle| genera la sección de título e información del autor, incluyendo el nombre del profesor y las afiliaciones. El título se puede modificar en lab-class/lab.cls/title style section.
    
    Por defecto, \textit{lab class} muestra el título a la izquierda. Sin embargo, puede cambiar \verb*|\raggedright| a \verb*|\centering| en \verb*|\titlepos| para mover el título al centro o modificarlo según sus preferencias.
    
    Además del comando \verb|\title|, se ha añadido un comando personalizado llamado \verb|\journalname| para incluir más información.
    
    Si no necesita este comando, puede indefinirlo y el contenido se ajustará automáticamente.

\section{Resumen}

    El resumen y las palabras clave se definen usando los comandos \verb*|\keywords| y \verb*|\begin{abstract} \end{abstract}| respectivamente. Para que aparezca el resumen, asegúrese de que el comando \verb|\lababstract| esté siempre incluido después del comienzo del documento.
    
    Si las palabras clave no se declaran en el preámbulo, el contenido se ajustará automáticamente.

\section{Tablas y figuras}

    \subsection{Tablas}
    
    La Tabla \ref{tab:ejemplo} muestra una tabla de ejemplo. El comando \verb|\tabletext{}| se usa para añadir notas a las tablas fácilmente.
    
    \begin{table}[H]
        \centering
        \caption{Ejemplo de tabla.}
        \label{tab:ejemplo}
        \begin{tabular}{ccc}
            \toprule
            Columna 1 & Columna 2 & Columna 3 \\
            \midrule
            1 & 2 & 3 \\
            4 & 5 & 6 \\
            7 & 8 & 9 \\
            \bottomrule
        \end{tabular}
        \tabletext{Esta es una nota de ejemplo para la tabla.}
    \end{table}

    \subsection{Figuras}
    
    La Figura \ref{fig:ejemplo} muestra un ejemplo de figura.
    
    \begin{figure}[H]
        \centering
        \fbox{\rule{0pt}{3cm}\rule{5cm}{0pt}}
        \caption{Ejemplo de figura.}
        \label{fig:ejemplo}
    \end{figure}

\section{Añadiendo códigos}

    Esta clase\footnote{¡Hola! Soy una nota al pie :)} incluye el paquete \textit{listings}, que ofrece características personalizadas para añadir códigos en documentos \LaTeX, específicamente para C, C++, \LaTeX\ y Matlab.
    
    Puede personalizar el formato en lab-class/lab.cls/listings style.
    
    \begin{lstlisting}[language=Matlab, caption={Ejemplo de código Matlab.}]
        % Ejemplo de código Matlab
        function y = ejemplo(x)
            y = sin(x) + cos(x);
        end
    \end{lstlisting}

%----------------------------------------------------------
% BIBLIOGRAPHY
%----------------------------------------------------------

\bibliographystyle{ieeetr}
\bibliography{lab}

\end{document}
