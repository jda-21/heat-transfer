% HEAT TRANSFER LAB TEMPLATE

\documentclass[9pt,a4paper,twocolumn,twoside]{lab-class/lab}

%----------------------------------------------------------
% LANGUAGE CONFIGURATION
%----------------------------------------------------------

% Uncomment the one you want to use
% \usepackage[english]{babel}
\usepackage[spanish,es-nodecimaldot,es-noindentfirst]{babel}
\usepackage{svg}

%----------------------------------------------------------
% DOCUMENT CONFIGURATION
%----------------------------------------------------------

% Editar datos del documento
% Título, autores, etc.

%----------------------------------------------------------
% TITLE
%----------------------------------------------------------

\title{Insertar título de informe}

%----------------------------------------------------------
% AUTHORS AND AFFILIATIONS
%----------------------------------------------------------

\author[1]{BRAYAN ARANGO GONZÁLEZ}
\author[2]{JUAN DIEGO ALVAREZ}
\author[3]{JULIÁN DAVID MARTÍNEZ}

\affil[1]{ID: 10038000 \\ brayan.arango@udea.edu.co}
\affil[2]{ID: 1003814249 \\ juan.alvarez67@udea.edu.co}
\affil[3]{ID: 123123123 \\ julian.martinezt@udea.edu.co}

\professor{ANDRES FELIPE COLORADO, JOEY DANIEL CASTAÑO}

%----------------------------------------------------------
% DOCUMENT INFORMATION
%----------------------------------------------------------

\institution{UNIVERSIDAD DE ANTIOQUIA}
\course{TRANSFERENCIA DE CALOR Y MASA}
\journalname{Plantilla de ejemplo}
\theday{Abril, 2025}


%----------------------------------------------------------
% ABSTRACT
%----------------------------------------------------------

\begin{abstract}    
    Insertar texto de referencia.
\end{abstract}

\keywords{Insertar palabras claves}

%----------------------------------------------------------

\begin{document}
		
    \maketitle 
    \thispagestyle{firststyle} 
    \lababstract 
    % \tableofcontents
    % \linenumbers 
    
%----------------------------------------------------------

\section{Resumen}
    % Breve descripción del experimento, objetivos principales y resultados más relevantes
    % Longitud recomendada: 150-250 palabras
    % Debe ser autónomo y dar una idea clara del trabajo realizado

    En este experimento se determinó la conductividad térmica de una muestra de aluminio utilizando el método de conducción unidimensional en estado estacionario. Se midió el gradiente de temperatura a lo largo de la muestra y el flujo de calor aplicado. Los resultados mostraron una conductividad térmica de 237 ± 5 W/m·K, que concuerda con los valores reportados en la literatura.

\section{Introducción}
    % Contextualización del experimento
    % Importancia y aplicaciones prácticas
    % Breve descripción de trabajos previos relevantes
    % Planteamiento del problema a resolver

    La determinación precisa de la conductividad térmica de los materiales es fundamental en diversas aplicaciones de ingeniería, desde el diseño de intercambiadores de calor hasta la optimización de sistemas de aislamiento térmico. Este experimento se centra en la medición de esta propiedad mediante un método experimental directo.
    
    Estudios previos han demostrado la importancia de considerar factores como la resistencia térmica de contacto y las pérdidas de calor al ambiente en este tipo de mediciones \cite{PFGPlots}.

\section{Objetivos}
    % Objetivo general del experimento
    % Objetivos específicos y mediciones a realizar
    % Resultados esperados

    \textbf{Objetivo General:}
    Determinar experimentalmente la conductividad térmica de una muestra de aluminio.

    \textbf{Objetivos Específicos:}
    \begin{itemize}
        \item Medir el gradiente de temperatura en estado estacionario
        \item Calcular el flujo de calor a través de la muestra
        \item Evaluar la resistencia térmica de contacto
    \end{itemize}

\section{Marco Teórico}
    \subsection{Conducción de Calor}
        % Principios básicos de la conducción de calor
        % Ecuaciones fundamentales
        % Variables relevantes

        La conducción de calor es el mecanismo de transferencia de energía térmica entre partículas adyacentes en un medio debido a un gradiente de temperatura. Este proceso es particularmente importante en sólidos, donde las moléculas oscilan alrededor de posiciones fijas.

    \subsection{Ley de Fourier}
        % Enunciado de la ley
        % Ecuaciones matemáticas
        % Aplicación al experimento

        La ley de Fourier establece que el flujo de calor por conducción es proporcional al gradiente de temperatura:
        \begin{equation}
            q = -k \frac{dT}{dx}
        \end{equation}
        donde $k$ es la conductividad térmica del material.

    \subsection{Resistencia Térmica de Contacto}
        % Definición y causas
        % Factores que la afectan
        % Importancia en el experimento

        La resistencia térmica de contacto surge debido a las imperfecciones microscópicas en las superficies de contacto entre materiales. Esta resistencia puede causar caídas de temperatura significativas en las interfaces.

\section{Metodología Experimental}
    \subsection{Materiales y Equipos}
        % Lista detallada de materiales utilizados
        % Especificaciones técnicas de los equipos
        % Instrumentos de medición y su precisión

        Los materiales y equipos utilizados en este experimento incluyen:
        \begin{itemize}
            \item Muestra de aluminio (10 cm x 2 cm x 2 cm)
            \item Fuente de calor eléctrica (0-100 W)
            \item 4 termopares tipo K (precisión: ± 0.1°C)
            \item Sistema de adquisición de datos
            \item Pasta térmica conductiva
        \end{itemize}

    \subsection{Descripción del Montaje}
        % Diagrama del montaje experimental
        % Explicación de cada componente
        % Conexiones y configuración

        El montaje experimental consiste en una muestra de aluminio instrumentada con termopares equidistantes. En un extremo se coloca la fuente de calor eléctrica, mientras que el otro extremo se mantiene a temperatura ambiente. La pasta térmica se aplica en todas las interfaces para minimizar la resistencia de contacto.

    \subsection{Procedimiento Experimental}
        % Pasos detallados del experimento
        % Variables medidas y controladas
        % Precauciones y consideraciones

        El procedimiento seguido fue:
        \begin{enumerate}
            \item Aplicar pasta térmica en las interfaces
            \item Colocar los termopares en las posiciones marcadas
            \item Configurar la fuente de calor a 50 W
            \item Esperar hasta alcanzar el estado estacionario ($\approx$ 30 min)
            \item Registrar las temperaturas cada 10 segundos durante 5 minutos
        \end{enumerate}

\section{Resultados y Análisis}
    \subsection{Datos Experimentales}
        % Tablas de datos recolectados
        % Condiciones experimentales
        % Observaciones durante el experimento

        Las mediciones de temperatura en estado estacionario se muestran en la Tabla \ref{tab:temperaturas}. Las condiciones ambientales durante el experimento fueron: temperatura ambiente de 23°C y humedad relativa del 45\%.

        \begin{table}[H]
            \centering
            \caption{Temperaturas medidas en estado estacionario}
            \label{tab:temperaturas}
            \begin{tabular}{cc}
                \toprule
                \textbf{Posición (cm)} & \textbf{Temperatura (°C)} \\
                \midrule
                0 & 85.3 ± 0.1 \\
                2 & 75.2 ± 0.1 \\
                4 & 65.1 ± 0.1 \\
                6 & 55.0 ± 0.1 \\
                \bottomrule
            \end{tabular}
        \end{table}

    \subsection{Gráficas y Ajustes}
        % Gráficas de los datos experimentales
        % Ajustes lineales o no lineales
        % Interpretación de las tendencias

        \begin{figure}[H]
            \centering
            \includegraphics[width=0.75\columnwidth]{Example.pdf}
            \caption{Perfil de temperatura a lo largo de la muestra.}
            \label{fig:gradiente}
        \end{figure}

        La Fig. \ref{fig:gradiente} muestra el perfil de temperatura a lo largo de la muestra. Se observa una relación lineal que confirma la validez de la ley de Fourier en estado estacionario.

    \subsection{Cálculo de Conductividad Térmica}
        % Metodología de cálculo
        % Resultados obtenidos
        % Comparación con valores teóricos

        La conductividad térmica se calculó usando la ecuación:
        \begin{equation}
            k = -\frac{q L}{A \Delta T}
        \end{equation}
        donde $q$ es el flujo de calor, $L$ la longitud, $A$ el área transversal y $\Delta T$ la diferencia de temperatura.

    \subsection{Análisis de Errores}
        % Fuentes de error
        % Cálculos de incertidumbre
        % Propagación de errores

        Las principales fuentes de error identificadas fueron:
        \begin{itemize}
            \item Incertidumbre en las mediciones de temperatura ($\pm 0.1^{\circ}$C)
            \item Variaciones en la potencia de la fuente ($\pm 1\%$)
            \item Pérdidas de calor al ambiente
        \end{itemize}

    \subsection{Resistencia Térmica de Contacto}
        % Cálculos realizados
        % Resultados obtenidos
        % Análisis de factores influyentes

        La resistencia térmica de contacto se estimó en $5.2 \times 10^{-6}$ m²K/W, lo cual es consistente con valores típicos para interfaces metal-metal con pasta térmica.

\section{Balance de Energía (opcional)}
    \subsection{Pérdidas por Convección y Radiación}
        % Estimación de pérdidas
        % Cálculos realizados
        % Impacto en los resultados

        Las pérdidas de calor al ambiente se estimaron considerando convección natural y radiación. Para la convección natural:
        \begin{equation}
            q_{conv} = h A (T_s - T_{\infty})
        \end{equation}
        donde $h \approx 5$ W/m²K es el coeficiente de convección natural promedio.

    \subsection{Cálculo de Temperatura del Fluido}
        % Metodología de cálculo
        % Resultados obtenidos
        % Análisis de implicaciones

        La temperatura del fluido circundante se modeló usando la correlación de Churchill y Chu para convección natural en placas verticales. Los resultados indican una capa límite térmica de aproximadamente 3 mm de espesor.

\section{Conclusiones}
    % Resumen de resultados principales
    % Comparación con objetivos planteados
    % Recomendaciones para futuros experimentos

    Los resultados principales de este experimento son:
    \begin{itemize}
        \item Se determinó la conductividad térmica del aluminio: 237 ± 5 W/m·K
        \item La resistencia térmica de contacto fue minimizada efectivamente usando pasta térmica
        \item Las pérdidas por convección y radiación representaron menos del 3\% del flujo de calor total
    \end{itemize}

    Para futuros experimentos se recomienda:
    \begin{itemize}
        \item Mejorar el aislamiento térmico para reducir pérdidas
        \item Aumentar el número de puntos de medición de temperatura
        \item Realizar mediciones a diferentes potencias de entrada
    \end{itemize}

\section{Bibliografía}
    % Referencias utilizadas en el marco teórico
    % Manuales de equipos consultados
    % Artículos científicos relevantes

    % CONFIGURACIÓN DE LA BIBLIOGRAFÍA:
    % 1. Las referencias se agregan al archivo lab.bib
    % 2. Para citar use \cite{clave_referencia} en el texto
    % 3. El estilo de citación se configura en lab.cls
    % 4. Para agregar referencias manualmente use el formato:
    %    @article{clave,
    %      author  = {Apellido, Nombre},
    %      title   = {Título del artículo},
    %      journal = {Nombre de la revista},
    %      year    = {año},
    %      volume  = {volumen},
    %      pages   = {páginas}
    %    }

    \printbibliography[heading=bibintoc]
    
\clearpage
% Incluir el manual de usuario
% MANUAL DE USUARIO - EJEMPLOS DE USO
% Este archivo contiene ejemplos de uso del template

\section*{MANUAL DE USUARIO - Ejemplos de uso}
\addcontentsline{toc}{section}{MANUAL DE USUARIO - Ejemplos de uso}

A continuación se presentan diversos ejemplos detallados que ilustran las múltiples posibilidades para enriquecer el contenido, incluyendo la incorporación de distintas secciones organizativas, 
la implementación de ecuaciones matemáticas complejas, la aplicación de variados formatos de texto y estilo, la inclusión de anotaciones y comentarios relevantes, la integración de fragmentos 
de código funcional, así como otros elementos y aspectos adicionales que pueden resultar sumamente beneficiosos y prácticos para mejorar la experiencia del usuario final.

\section*{Tablas y figuras}
    % This section demonstrates how to use tables and figures in your document

    \subsection*{Tablas}
        % Example of a simple table with notes
        \begin{table}[H]
            \centering
            \caption{Astronomical Object Data}
            \label{tab:table}
            \begin{tabular}{ll}
                \toprule
                \textbf{Object} & \textbf{Distance (Light Years)} \\
                \midrule
                Alpha Centauri & 4.37 \\
                Betelgeuse & 642.5 \\
                Andromeda Galaxy & 2.537 million \\
                \bottomrule   
            \end{tabular}
            \tabletext{Note: The table contains data of some famous celestial objects.}
        \end{table}

        The table above (Table \ref{tab:table}) demonstrates the use of \verb|\tabletext{}| for adding notes to tables. 

    \subsection*{Figuras}
        % Example of a basic figure
        \begin{figure}[H]
            \centering
            \includegraphics[width=0.6\columnwidth]{Example.pdf}
            \caption{A simple figure example using PGFPlots \cite{PGFPlots}.}
            \label{fig:figure}
        \end{figure}

        The figure above (Fig. \ref{fig:figure}) shows a basic example of including a figure. For more complex layouts, you can use subfigures as shown below.

        % Example of subfigures
        \begin{figure}[H]
            \centering
            \begin{subfigure}[b]{0.45\textwidth}
                \centering
                \includegraphics[width=\linewidth]{Example2.pdf}
                \caption{Left subfigure example}
                \label{fig:figa}
            \end{subfigure}
            \hspace{0.05\textwidth}
            \begin{subfigure}[b]{0.45\textwidth}
                \centering
                \includegraphics[width=\linewidth]{Example3.pdf}
                \caption{Right subfigure example}
                \label{fig:figb}
            \end{subfigure}
            \caption{Example of multiple figures using the subfigure environment}
            \label{fig:examplefloat}
        \end{figure}

        The figure above (Fig. \ref{fig:examplefloat}) demonstrates how to arrange multiple subfigures. You can adjust the spacing between subfigures using \verb|\hspace{...}| and their relative sizes using the width parameter.
		
\section*{Paquetes Lab}

    \subsection*{Tauenvs}
	
        This template has its own environment package \textit{labenvs.sty} designed to enhance the presentation of the document. Among these custom environments are \textit{labenv}, \textit{info} and \textit{note}.
		
        There are two environments which have a predefined title. These can be included by the command \verb|\begin{note}| and \verb|\begin{info}|. All the environments have the same style.
			
        An example using the tau environment is shown below.
		
    	\begin{labenv}[frametitle=Environment with custom title]
            This is an example of the custom title environment. To add a title type \verb|[frametitle=Your title]| next to the beginning of the environment (as shown in this example).
    	\end{labenv}
		
        Tauenv is the only environment that you can customize its title. On the other hand, info and note adapt their title to Spanish automatically when this language package is defined.
		
    \subsection*{Taubabel}

        In previous versions, we included a package called \textit{lab-babel}, which have all the commands that automatically translate from English to Spanish when this language package is defined. 
        
        By default, tau displays its content in English. However, at the beginning of the document you will find a recommendation when writing in Spanish. 
		
        \textit{Note:} You may modify this package if you want to use other language than English or Spanish. This will make easier to translate the document without having to modify the class document.
		
\section*{Ecuación}

    Equation \ref{ec:equation}, shows the Schrödinger equation as an example. 
	\begin{equation} \label{ec:equation}
		\frac{\hbar^2}{2m}\nabla^2\Psi + V(\mathbf{r})\Psi = -i\hbar \frac{\partial\Psi}{\partial t}
	\end{equation} 
    The \textit{amssymb} package was not necessary to include, because stix2 font incorporates mathematical symbols for writing quality equations. In case you choose another font, uncomment this package in lab-class/tau.cls/math packages.
	
    If you want to change the values that adjust the spacing above and below the equations, play with \verb|\setlength{\eqskip}{8pt}| value until the preferred spacing is set.
	

\section*{Apéndice}

    \subsection*{Título alternativo}

        You can make the following modification in lab-class/tau.cls/title preferences section to change the position of the title.

\begin{lstlisting}[language=TeX, caption=Alternative title.]
\newcommand{\titlepos}{\centering}
\end{lstlisting}

        This will move the title to the center. 

    \subsection*{Entorno de información}

        An example of the info environment declared in the 'labenvs.sty' package is shown below. Remember that \textit{info} and \textit{note} are the only packages that translate their title (English or Spanish).
		
	\begin{info}
		Small example of info environment.
	\end{info}

    \subsection{Equation skip value}

        With the \verb|\eqskip| command you can change the spacing for equations. The default \textit{eqskip} value is 8pt.

\begin{lstlisting}[language=TeX, caption=Equation skip code.]
\newlength{\eqskip}\setlength{\eqskip}{8pt}
	\expandafter\def\expandafter\normalsize\expandafter{%
		\normalsize%
		\setlength\abovedisplayskip{\eqskip}%
		\setlength\belowdisplayskip{\eqskip}%
		\setlength\abovedisplayshortskip{\eqskip-\baselineskip}%
		\setlength\belowdisplayshortskip{\eqskip}%
\end{lstlisting}

		
    \subsection{References}
		
        In case you require another reference style, you can go to lab-class/tau.cls/biblatex and modify the following.
		
\begin{lstlisting}[language=TeX, caption=References style.]
\RequirePackage[
	backend=biber,
	style=ieee,
	sorting=ynt
]{biblatex}
\end{lstlisting}

        By default, \textit{lab class} has its own .bib for this example, if you want to name your own bib file, change the \textit{addbibresource}.
		
\begin{lstlisting}[language=TeX]
\addbibresource{tau.bib}
\end{lstlisting}

\section{Uso de notas}

    \subsection*{How do I manage my references?}
        
        To manage your references, I recommend using the tool \href{https://www.scribbr.es/citar/generador/folders/73QOXYsCwMRu4ifQaN65mx/lists/msTfx7GJjIAOUkufbISnA/}{scribbr}. You can simply enter the URL or create your own citation, and then export it to \LaTeX\ using the options in the three-dot menu.
            
        The generated citation can be copied and pasted into \textit{tau.bib}, the file designated for bibliography management. You may rename this file, but if you do, remember to update the \verb|\addbibresource| command in \textit{tau.cls} under the \textit{biblatex} section.
    
        \begin{note}
            Some platforms, such as Google Scholar or scientific journals, provide citations directly in \LaTeX\ format. Therefore, check if there is a ``how to cite this document'' section to streamline the citation process even further.
        \end{note}
    
        Here is an example of a reference code compatible with Bib\TeX.

\begin{verbatim}
@misc{PFGPlots,
    author = {PFGPlots},
    title = {A LaTeX package to create plots.},
    url = {https://pgfplots.sourceforge.net/}
}
\end{verbatim}


    
%----------------------------------------------------------

%----------------------------------------------------------

\end{document}